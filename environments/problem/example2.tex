\documentclass{article}
\usepackage{amsmath, amsfonts}
\usepackage{mathtools}
\usepackage{parskip}
\usepackage[explicit]{titlesec}
\usepackage[table, dvipsnames]{xcolor}
\usepackage{colortbl}
\usepackage{asymptote}
\newcounter{problem}[section]
\newenvironment{problem}
    {
    \newpage
    \refstepcounter{problem}\par\medskip\textbf{Problem \theproblem:}\medskip\rmfamily \newline
    \rowcolors{1}{Yellow}{Yellow}
    \begin{tabular}{|p{0.9\textwidth}|}
    \hline\\
    }
    { 
    \\\\\hline
    \end{tabular}
    }
\begin{document}
\begin{problem}
Three corners of a parallelogram are shown below:
\medskip
\begin{center}
\begin{asy}
settings.outformat="pdf";
defaultpen(fontsize(8pt));
unitsize(.5cm);
Label Lx= Label("$x$", position=EndPoint, align=Relative(E));
Label Ly= Label("$y$", position=EndPoint, align=Relative(E));
draw((-6,0) -- (6,0), arrow=Arrow(TeXHead), L=Lx);
draw((0,-6) -- (0,6), arrow=Arrow(TeXHead), L=Ly);
draw((1, -0.15) -- (1, 0.15), L= Label("1", position=BeginPoint));
draw((-1, -0.15) -- (-1, 0.15), L= Label("-1", position=BeginPoint));
draw((3, -0.15) -- (3, 0.15), L= Label("3", position=BeginPoint));
draw((-3, -0.15) -- (-3, 0.15), L= Label("-3", position=BeginPoint));
draw((5, -0.15) -- (5, 0.15), L= Label("5", position=BeginPoint));
draw((-5, -0.15) -- (-5, 0.15), L= Label("-5", position=BeginPoint));
draw((-0.15, 1) -- (.15, 1), L= Label("1", position=BeginPoint));
draw((-0.15, -1) -- (.15, -1), L= Label("-1", position=BeginPoint));
draw((-0.15, 3) -- (.15, 3), L= Label("3", position=BeginPoint));
draw((-0.15, -3) -- (.15, -3), L= Label("-3", position=BeginPoint));
draw((-0.15, 5) -- (.15, 5), L= Label("5", position=BeginPoint));
draw((-0.15, -5) -- (.15, -5), L= Label("-5", position=BeginPoint));
pen pointsPen = 5+black;
dot((0, 0), pointsPen);
dot((2, 4), pointsPen);
dot((0, 3), pointsPen);
\end{asy}
\end{center}
\medskip
What is the fourth corner?
\end{problem}

\end{document}
